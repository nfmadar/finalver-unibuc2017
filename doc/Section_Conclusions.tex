\section{Concluzii}
\paragraph{}

Aceast;a lucrare a fost o prezentare a unui model de re'tele neuro-fuzzy, 'si anume ANFIS, c'at 'si aplicarea acestuia asupra unui set de date bazat pe COCOMO. Pentru a ob'tine o viziune mai bun;a asupra performan'telor acestui model, rezultatele acestuia au fost comparate cu cele ob'tinute dintr-o re'tea neuronal;a clasic;a, fiind un model cunoscut 'si bine studiat.
\par
Rezultatele ini'tiale ne arat;a faptul c;a ANFIS se descurc;a foarte bine, cel pu'tin pentru acest set de date, ob'tin'and o predic'tie cu eroare sub 25\% pentru toate proiectele repartizate 'in mul'timea de testare. 'In cealalt;a parte, modelul bazat pe re'tele neuronale ob'tine aceea'si predic'tie doar 'in propor'tie de 83\%, fiind aproape la limita de a fi un predictor acceptabil.
\par
Consider;am c;a unul dintre cele mai puternice puncte ale acestei lucr;ari a fost 'inv;a'tarea peste toate cele 16 caracteristici ale modelului COCOMO, fa't;a de alte articole din literatur;a care se restr'ang la un num;ar mai mic de caracteristici. 'In plus, ANFIS poate fi 'in mod manual ajustat cu cuno'stin'te expert, fiind p'an;a la urm;a un sistem de inferen't;a fuzzy. Acest lucru permite 'si analizarea foarte u'soar;a 'in cazul 'in care estim;arile au erori foarte mari, fiind posibil;a g;asirea regulii 'si/sau a clusterului care genereaz;a aceste defecte.
\par
O viitoare direc'tie de cercetare poate consta 'in a g;asi metode c'at mai eficiente de integrare a cuno'stin'telor expert 'inc;a din momentul antren;arii re'telelor ANFIS, astfel 'imbun;at;a'tind eficien;ta 'inv;a'tarii. 'In plus, compara'tia ar putea fi reluat;a 'si 'imbun;at;a'tit;a at'at pentru re'telele ANFIS, c'at 'si pentru cele neuronale dac;a ar exista o metod;a verosimil;a de a genera mul'timi noi de proiecte sintetice sub modelul COCOMO.
\par
'In concluzie, noi consider;am modelul ANFIS ca av'and un domeniu de aplica'tie foarte bun 'in predic'tia efortului de dezvoltare ale proiectelor software. 'In plus, p;astr'and aceea'si arhitectur;a a re'telei, scopul acesteia poate fi schimbat pentru a prezice viitoarele defecte dintr-un proiect software (desigur, date fiind datele de antrenare corespunz;atoare) sau puncte 'in seriile de timp. Aplicabilitatea ANFIS a fost deja extins;a cu succes 'si in domeniul controlului automat, 'si cu siguran't;a acesta este doar 'inceputul.