\section {Mediul de programare}

\paragraph{}

\par
Pentru implementarea modelelor predictive ce vor fi prezentate 'in aceast;a lucrare, mediul de dezvoltare este Matlab.
\par
Matlab este un limbaj de programare interpretat multi-paradigm;a, centrat 'in jurul manipul;arii matricelor, cu tipuri de date slabe (variabilele sunt convertite implicit) 'si inferate (variabilele pot avea valori atribuite fara ca tipul lor s;a fie declarat).  'In plus, cei de la MathWorks ofer;a o mul'time de biblioteci (denumite \textit{toolboxuri}) pentru diverse aplica'tii 'in matematic;a numeric;a 'si simbolic;a, procesare de semnale 'si imagini, sisteme de control etc. De asemenea, exist;a suport pentru exportarea codului 'in limbaje de programare tradi'tionale precum C, Java sau Python, dar 'si pentru apelarea de func'tii scrise 'in aceste limbaje folosind un \textit{wrapper} care permite tipurilor de date din MATLAB s'a fie pasate 'si primite 'in mod dinamic.  Prin urmare, am considerat c;a Matlab ofer;a uneltele necesare pentru a dezvolta o solu'tie orientat;a pe rezultate, 'si nu pe detaliile de implementare la nivel de limbaj de programare. 
\par
Aplica'tiile lucr;arii s-au bazat pe toolboxurile \textit{Fuzzy Logic Toolbox} 'si \textit{Neural Network Toolbox}. Primul toolbox ne-a ajutat s;a gener;am un sistem de inferen't;a fuzzy de tip Sugeno, s;a clusteriz;am datele de intrare 'si s;a putem construi o re'tea ANFIS. Cel de-al doilea ne-a oferit posibilitatea de a construi 'si antrena o re'tea neuronal;a folosit;a drept compara'tie pentru modelul neuro-fuzzy propus. 